\documentclass[12pt]{article}

\usepackage[utf8]{inputenc}
\usepackage[spanish]{babel}
\usepackage{enumerate}
\usepackage{amsmath}
\usepackage{txfonts}
\usepackage{graphicx}
\usepackage{keystroke}
\usepackage{color}
\usepackage[colorlinks = true,
linkcolor = green,
urlcolor = blue,
citecolor = yellow,
anchorcolor = brown]{hyperref}
\usepackage{listings}
\usepackage{alltt}
\usepackage{multicol}

\definecolor{mygreen}{rgb}{0,0.6,0}
\definecolor{mygray}{rgb}{0.5,0.5,0.5}
\definecolor{mymauve}{rgb}{0.58,0,0.82}

\lstset{ %
  backgroundcolor=\color{white},   % choose the background color; you must add \usepackage{color} or \usepackage{xcolor}
  basicstyle=\footnotesize,        % the size of the fonts that are used for the code
  breakatwhitespace=false,         % sets if automatic breaks should only happen at whitespace
  breaklines=true,                 % sets automatic line breaking
  captionpos=b,                    % sets the caption-position to bottom
  commentstyle=\color{mygreen},    % comment style
  deletekeywords={...},            % if you want to delete keywords from the given language
  escapeinside={\%*}{*)},          % if you want to add LaTeX within your code
  extendedchars=true,              % lets you use non-ASCII characters; for 8-bits encodings only, does not work with UTF-8
  frame=single,                    % adds a frame around the code
  keepspaces=true,                 % keeps spaces in text, useful for keeping indentation of code (possibly needs columns=flexible)
  keywordstyle=\color{blue},       % keyword style
  language=Octave,                 % the language of the code
  morekeywords={*,...},            % if you want to add more keywords to the set
  numbers=left,                    % where to put the line-numbers; possible values are (none, left, right)
  numbersep=5pt,                   % how far the line-numbers are from the code
  numberstyle=\tiny\color{mygray}, % the style that is used for the line-numbers
  rulecolor=\color{black},         % if not set, the frame-color may be changed on line-breaks within not-black text (e.g. comments (green here))
  showspaces=false,                % show spaces everywhere adding particular underscores; it overrides 'showstringspaces'
  showstringspaces=false,          % underline spaces within strings only
  showtabs=false,                  % show tabs within strings adding particular underscores
  stepnumber=2,                    % the step between two line-numbers. If it's 1, each line will be numbered
  stringstyle=\color{mymauve},     % string literal style
  tabsize=2,                       % sets default tabsize to 2 spaces
  title=\lstname                   % show the filename of files included with \lstinputlisting; also try caption instead of title
}

\newcommand{\noterm}[1]{\color{blue} \langle #1 \rangle \color{blue}}
\newcommand{\produce}{\color{red} \coloneqq \color{red}}
\newcommand{\alter}{\color{red} \mid \color{red}}
\newcommand{\catterm}[1]{\color{magenta} \text{#1} \color{magenta}}
\newcommand{\term}[1]{\color{cyan} \text{#1} \color{cyan}}
\newcommand{\openopt}{\color{black} \text{[} \color{black}}
\newcommand{\closeopt}{\color{black} \text{]} \color{black}}
\newcommand{\severals}{\color{black} \ldots \color{black}}

\newcounter{ejercicio}
\newcommand{\ejercicio}{\stepcounter{ejercicio}%
  \paragraph\noindent\textbf{Ejercicio \theejercicio.\hspace{4pt}}}

\newcounter{problema}
\newcommand{\problema}{\stepcounter{problema}%
  \paragraph\noindent\textbf{Problema \theproblema.\hspace{4pt}}}
\newcommand{\WindowsLogo}{\raisebox{-0.1em}{%
    \includegraphics[height=0.8em]{../../imagenes/Windows_3_logo_simplified}}}
\newcommand{\WinKey}{\keystroke{\WindowsLogo}}

\newcommand{\Subversion}{\href{https://subversion.apache.org/}{subversion}}
\newcommand{\Riouxsvn}{\href{https://riouxsvn.com/}{Riouxsvn}}

\title{Taller de Funciones y recursión en Scala}
\date{24 de febrero de 2021}
\author{S4N Campus}
%\institute{S4N}

\begin{document}
\maketitle

\section{Preliminares}
\label{sec:preliminares}

Este taller cuenta con el objetivo de entender que es la programación
funcional en Scala.

\section{Funciones literales/Funciones anónimas}
\label{sec:fun-lit-fun-anon}

\ejercicio. Defina una \emph{función literal} llamada \texttt{areaTrianguloRectangulo} que se encargue de calcular el área de un triángulo rectángulo. Recibe los dos lados rectos.
\ejercicio. Defina una \emph{función literal} llamada \texttt{areaDeUnCirculo} que en vez de utilizar una función anónima, utilice la implementación del \texttt{trait} correspondiente (\texttt{Function1}, \texttt{Function2}, \texttt{Funtion3}, $\ldots$).

\section{Funciones de primera clase y funciones de alto orden}

\ejercicio Defina una función literal llamada \texttt{calSalario} que reciba dos parámetros de tipo \texttt{Double} (devengado y deducciones) y
que retorna el valor clásico del salario de una persona:
\[
  devengado - deducciones
\]
\ejercicio Defina una función literal llamada \texttt{calSalarioBono} que reciba dos parámetros de tipo \texttt{Double} (devengado y deducciones) y calcule el salario con el siguiente valor.
\[
  devengado * 1.10 - deducciones
\]
\ejercicio Defina una función llamada \texttt{compSalario} que recibe tres parámetros: el primero una función de tipo \texttt{(Double,Double)=>Double} y las otras dos son: \texttt{devengado} y \texttt{deducciones}. Prueba esta función pasado las dos
funciones anteriores \texttt{calSalario} y \texttt{calSalarioBono}.
\ejercicio Defina una función llamada \texttt{genCalSalarioBono} esta función se encarga de generar funciones que computan diferente bonos. La función

\begin{lstlisting}[language=Scala]
def genCalSalarioBono(bono:Double):(Double,Double) => Double = ???
\end{lstlisting}
\ejercicio Utilizando la función generadora de funciones \texttt{genCalSalarioBono} cree la función literal \texttt{calSalario5} que da un bono
del 5\%.
\ejercicio Utilizando la función generadora de funciones \texttt{genCalSalarioBono} cree la función literal \texttt{calCalario20} que da un bono
del 20\%.

\section{Clausuras}
\label{sec:clausuras}

\ejercicio Declare una función \texttt{calSalarioBonoClausura} que reciba \emph{dos parámetros} (devengados y deducciones) y que aplique la siguiente fórmula:

\[
  devengado \times bono - deducciones
\]

Donde \texttt{bono} es una valor declarado externamente (Una clausura).

\ejercicio Utilizando la función \texttt{compSalario} aplique la función utilizando como primer parámetro \texttt{calSalarioBonoClausura} y calculado varios salarios diferentes.

\section{Funciones aplicadas parcialmente}
\label{sec:func-aplic-parc}

\ejercicio Utilizando la función \texttt{genCalSalarioBono} cree una función literal \texttt{calCalario15} que se obtiene
a través de la aplicación parcial de \texttt{genCalSalarioBono} con un valor $0.15$
\ejercicio Utilizando la función \texttt{genCalSalarioBono} cree una función literal \texttt{calCalario100} que se obtiene
a través de la aplicación parcial de \texttt{genCalSalarioBono} con un valor $2.00$.

\section{Funciones currificadas}
\label{sec:func-curr}

\ejercicio Utilizando currificación defina una función \texttt{genCalSalarioBono2} donde el último parámetro (el currificado) es el valor del bono y los dos primeros parámetros son: el devengado y la deducción.
\ejercicio Utilizando la función \texttt{genCalSalarioBono2} cree una función literal \texttt{calCalario25} que se obtiene
a través de la aplicación currificada de \texttt{genCalSalarioBono2} con un valor $0.25$.

\pagebreak

\section{Recursividad}
\label{sec:recursividad}

\ejercicio Implementar la función de factorial utilizando recursividad en Scala.

\[
  factorial(n) = 
\begin{cases}
  1 & n = 0\\
  1 & n = 1\\
  n \times factorial(n-1) & \text{Otra forma}\\
\end{cases}
\]

\ejercicio Implementar la siguiente función de forma recursiva en Scala

\[
  f(n) =
  \begin{cases}
    0 & n = 0 \\
    1 & n = 1 \\
    f(n-1) + f(n-2) & \text{Otra forma}\\
  \end{cases}
\]

\ejercicio Reescriba la función de factorial para que se ejecute bajo recursividad de cola.

% \section{Bibliografía}
% \label{sec:bibliografia}
% \bibliographystyle{amsalpha}
% \bibliography{S4N_Git_Tutorial}

\end{document}

%%% Local Variables:
%%% mode: latex
%%% TeX-master: t
%%% End:
