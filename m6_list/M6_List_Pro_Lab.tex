\documentclass[12pt]{article}

\usepackage[utf8]{inputenc}
\usepackage[spanish]{babel}
\usepackage{enumerate}
\usepackage{amsmath}
\usepackage{txfonts}
\usepackage{graphicx}
\usepackage{keystroke}
\usepackage{color}
\usepackage[colorlinks = true,
linkcolor = green,
urlcolor = blue,
citecolor = yellow,
anchorcolor = brown]{hyperref}
\usepackage{listings}
\usepackage{alltt}
\usepackage{multicol}

\definecolor{mygreen}{rgb}{0,0.6,0}
\definecolor{mygray}{rgb}{0.5,0.5,0.5}
\definecolor{mymauve}{rgb}{0.58,0,0.82}

\lstset{ %
  backgroundcolor=\color{white},   % choose the background color; you must add \usepackage{color} or \usepackage{xcolor}
  basicstyle=\footnotesize,        % the size of the fonts that are used for the code
  breakatwhitespace=false,         % sets if automatic breaks should only happen at whitespace
  breaklines=true,                 % sets automatic line breaking
  captionpos=b,                    % sets the caption-position to bottom
  commentstyle=\color{mygreen},    % comment style
  deletekeywords={...},            % if you want to delete keywords from the given language
  escapeinside={\%*}{*)},          % if you want to add LaTeX within your code
  extendedchars=true,              % lets you use non-ASCII characters; for 8-bits encodings only, does not work with UTF-8
  frame=single,                    % adds a frame around the code
  keepspaces=true,                 % keeps spaces in text, useful for keeping indentation of code (possibly needs columns=flexible)
  keywordstyle=\color{blue},       % keyword style
  language=Octave,                 % the language of the code
  morekeywords={*,...},            % if you want to add more keywords to the set
  numbers=left,                    % where to put the line-numbers; possible values are (none, left, right)
  numbersep=5pt,                   % how far the line-numbers are from the code
  numberstyle=\tiny\color{mygray}, % the style that is used for the line-numbers
  rulecolor=\color{black},         % if not set, the frame-color may be changed on line-breaks within not-black text (e.g. comments (green here))
  showspaces=false,                % show spaces everywhere adding particular underscores; it overrides 'showstringspaces'
  showstringspaces=false,          % underline spaces within strings only
  showtabs=false,                  % show tabs within strings adding particular underscores
  stepnumber=2,                    % the step between two line-numbers. If it's 1, each line will be numbered
  stringstyle=\color{mymauve},     % string literal style
  tabsize=2,                       % sets default tabsize to 2 spaces
  title=\lstname                   % show the filename of files included with \lstinputlisting; also try caption instead of title
}

\newcommand{\noterm}[1]{\color{blue} \langle #1 \rangle \color{blue}}
\newcommand{\produce}{\color{red} \coloneqq \color{red}}
\newcommand{\alter}{\color{red} \mid \color{red}}
\newcommand{\catterm}[1]{\color{magenta} \text{#1} \color{magenta}}
\newcommand{\term}[1]{\color{cyan} \text{#1} \color{cyan}}
\newcommand{\openopt}{\color{black} \text{[} \color{black}}
\newcommand{\closeopt}{\color{black} \text{]} \color{black}}
\newcommand{\severals}{\color{black} \ldots \color{black}}

\newcounter{ejercicio}
\newcommand{\ejercicio}{\stepcounter{ejercicio}%
  \paragraph\noindent\textbf{Ejercicio \theejercicio.\hspace{4pt}}}

\newcounter{problema}
\newcommand{\problema}{\stepcounter{problema}%
  \paragraph\noindent\textbf{Problema \theproblema.\hspace{4pt}}}
\newcommand{\WindowsLogo}{\raisebox{-0.1em}{%
    \includegraphics[height=0.8em]{../../imagenes/Windows_3_logo_simplified}}}
\newcommand{\WinKey}{\keystroke{\WindowsLogo}}

\newcommand{\Subversion}{\href{https://subversion.apache.org/}{subversion}}
\newcommand{\Riouxsvn}{\href{https://riouxsvn.com/}{Riouxsvn}}

\title{\emph{List}s en Scala}
\date{4 de marzo de 2021}
\author{S4N Campus}
%\institute{S4N}

\begin{document}
\maketitle

\section{Preliminares}
\label{sec:preliminares}

Este taller tiene como objetivo poner en practica los elementos de
\emph{List} en Scala y sus operaciones.

\section{Listas}
\label{sec:listas}

\ejercicio La función \texttt{subs} me permite obtener todos los
subconjuntos que un conjunto tiene. En este ejemplo vamos a utilizar
listas como un conjunto.

\begin{lstlisting}[language=Scala]
def subs[A](lst:List[A]):List[List[A]] = ???
\end{lstlisting}

Por ejemplo (En las pruebas las listas no tendrá elementos repetidos).

\begin{alltt}
scala> subs(List(1,2))
val res0:List[List[Int]]=List(List(),List(1),List(2),List(1,2))
scala> subs(List("a","b","c"))
val res1:List[List[String]]=List(List(),List("a"),List("b"),List("c"),
                                 List("a","b"),List("a","c"),
                                 List("b","c"),List("c","a"),
                                 List("a","b","c"))
\end{alltt}

\ejercicio La permutación es la manera de combinar todos los posibles
elementos de una lista en un conjunto de listas con todas las
combinaciones posibles. En Scala el tipo de dato \texttt{List} tiene
implementado un método de como obtener dichas listas. Nosotros vamos a
implementar una función \texttt{permutaciones} que dada la siguiente
firma:

\begin{lstlisting}[language=Scala]
def permutaciones[A](lst:List[A]):List[List[A]] = ???
\end{lstlisting}

\emph{\textbf{Sin utilizar}} el método \texttt{permutations} en clase
alguna de Scala implementar la forma de computar todas las
permutaciones.

Por ejemplo:

\begin{alltt}
scala> permutaciones(List("a","b","c"))
res0:List[List[String]]=List(List("a","b","c"),List("b","a","c"),
                             List("b","c","a"),List("a","c","b"),
                             List("c","a","b"),List("c","b","a"))
scala> permutaciones(List(2,3,4))
res1:List[List[Int]]=List(List(2,3,4),List(3,2,4),
                          List(3,4,2),List(2,4,3),
                          List(4,2,3),List(4,3,2))
\end{alltt}

$\square$



\ejercicio Defina una función \texttt{predAtPos} que tiene la siguiente firma:

\begin{lstlisting}[language=Scala]
def predAtPos[A](lst:List[A],preds:List[(Int,p:A=>Boolean)]):List[[Boolean]] = ???
\end{lstlisting}

Esta función recibe dos parámetros \texttt{lst} y \texttt{preds}. En \texttt{preds} es una tupla donde el primer elemento y el segundo
es un predicado, el primer elemento tiene el índice de la \texttt{lst} donde debe aplicar el predicado. Cada aplicación de un \texttt{pred} produce una lista singular con el valor obtenido del predicado o lista vacía si el indice no es válido en la lista.

Por ejemplo:

{\scriptsize
\begin{alltt}
scala> predAtPos(List(1,2,3,4,5,6),List((4,_==5),(0,_%mod 2 == 1), (5,_%mod 2 == 1)))
res0:List[List[Boolean]] = List(List(true),List(true),List(false))
scala> predAtPos(List(1,2,3,4,5,6),List((1,_<=2),(6,_%mod 2 == 0),(0,_>0)))
res1:List[List[Boolean]] = List(List(true),List(),List(true))
\end{alltt}
}

$\square$

\ejercicio Implemente la función \texttt{circulaI} que tiene la siguiente firma:

\begin{lstlisting}[language=Scala]
def circulaI[A](lst:List[A]):Lista[A] = ???
\end{lstlisting}

Esta función se encarga permuta circularmente de izquierda la lista \texttt{lst}

Por ejemplo:

\begin{alltt}
scala> circulaI(List(1,2,3,4,5))
res0:List[Int] = List[2,3,4,5,1]
scala> circulaI(List("a"))
res1:List[String] = List("a")
scala> circulaI(List(true,false,true))
res2:List[Boolean] = List(false,true,true)
\end{alltt}


\ejercicio Implemente la función \texttt{circulaD} que tiene la siguiente firma:

\begin{lstlisting}[language=Scala]
def circulaD[A](lst:List[A]):Lista[A] = ???
\end{lstlisting}

Esta función se encarga permuta circularmente de derecha la lista \texttt{lst}

Por ejemplo:

\begin{alltt}
scala> circulaD(List(1,2,3,4,5))
res0:List[Int] = List[5,1,2,3,4,5]
scala> circulaD(List("a"))
res1:List[String] = List("a")
scala> circulaD(List(true,false,true))
res2:List[Boolean] = List(true,true,false)
\end{alltt}

\ejercicio 

% \section{Bibliografía}
% \label{sec:bibliografia}
% \bibliographystyle{amsalpha}
% \bibliography{S4N_Git_Tutorial}

\end{document}

%%% Local Variables:
%%% mode: latex
%%% TeX-master: t
%%% End:
