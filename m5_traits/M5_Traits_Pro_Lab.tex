\documentclass[12pt]{article}

\usepackage[utf8]{inputenc}
\usepackage[spanish]{babel}
\usepackage{enumerate}
\usepackage{amsmath}
\usepackage{txfonts}
\usepackage{graphicx}
\usepackage{keystroke}
\usepackage{color}
\usepackage[colorlinks = true,
linkcolor = green,
urlcolor = blue,
citecolor = yellow,
anchorcolor = brown]{hyperref}
\usepackage{listings}
\usepackage{alltt}
\usepackage{multicol}

\definecolor{mygreen}{rgb}{0,0.6,0}
\definecolor{mygray}{rgb}{0.5,0.5,0.5}
\definecolor{mymauve}{rgb}{0.58,0,0.82}

\lstset{ %
  backgroundcolor=\color{white},   % choose the background color; you must add \usepackage{color} or \usepackage{xcolor}
  basicstyle=\footnotesize,        % the size of the fonts that are used for the code
  breakatwhitespace=false,         % sets if automatic breaks should only happen at whitespace
  breaklines=true,                 % sets automatic line breaking
  captionpos=b,                    % sets the caption-position to bottom
  commentstyle=\color{mygreen},    % comment style
  deletekeywords={...},            % if you want to delete keywords from the given language
  escapeinside={\%*}{*)},          % if you want to add LaTeX within your code
  extendedchars=true,              % lets you use non-ASCII characters; for 8-bits encodings only, does not work with UTF-8
  frame=single,                    % adds a frame around the code
  keepspaces=true,                 % keeps spaces in text, useful for keeping indentation of code (possibly needs columns=flexible)
  keywordstyle=\color{blue},       % keyword style
  language=Octave,                 % the language of the code
  morekeywords={*,...},            % if you want to add more keywords to the set
  numbers=left,                    % where to put the line-numbers; possible values are (none, left, right)
  numbersep=5pt,                   % how far the line-numbers are from the code
  numberstyle=\tiny\color{mygray}, % the style that is used for the line-numbers
  rulecolor=\color{black},         % if not set, the frame-color may be changed on line-breaks within not-black text (e.g. comments (green here))
  showspaces=false,                % show spaces everywhere adding particular underscores; it overrides 'showstringspaces'
  showstringspaces=false,          % underline spaces within strings only
  showtabs=false,                  % show tabs within strings adding particular underscores
  stepnumber=2,                    % the step between two line-numbers. If it's 1, each line will be numbered
  stringstyle=\color{mymauve},     % string literal style
  tabsize=2,                       % sets default tabsize to 2 spaces
  title=\lstname                   % show the filename of files included with \lstinputlisting; also try caption instead of title
}

\newcommand{\noterm}[1]{\color{blue} \langle #1 \rangle \color{blue}}
\newcommand{\produce}{\color{red} \coloneqq \color{red}}
\newcommand{\alter}{\color{red} \mid \color{red}}
\newcommand{\catterm}[1]{\color{magenta} \text{#1} \color{magenta}}
\newcommand{\term}[1]{\color{cyan} \text{#1} \color{cyan}}
\newcommand{\openopt}{\color{black} \text{[} \color{black}}
\newcommand{\closeopt}{\color{black} \text{]} \color{black}}
\newcommand{\severals}{\color{black} \ldots \color{black}}

\newcounter{ejercicio}
\newcommand{\ejercicio}{\stepcounter{ejercicio}%
  \paragraph\noindent\textbf{Ejercicio \theejercicio.\hspace{4pt}}}

\newcounter{problema}
\newcommand{\problema}{\stepcounter{problema}%
  \paragraph\noindent\textbf{Problema \theproblema.\hspace{4pt}}}
\newcommand{\WindowsLogo}{\raisebox{-0.1em}{%
    \includegraphics[height=0.8em]{../../imagenes/Windows_3_logo_simplified}}}
\newcommand{\WinKey}{\keystroke{\WindowsLogo}}

\newcommand{\Subversion}{\href{https://subversion.apache.org/}{subversion}}
\newcommand{\Riouxsvn}{\href{https://riouxsvn.com/}{Riouxsvn}}

\title{\emph{Traits} en Scala}
\date{3 de marzo de 2021}
\author{S4N Campus}
%\institute{S4N}

\begin{document}
\maketitle

\section{Preliminares}
\label{sec:preliminares}

Este taller tiene como objetivo poner en practica los elementos de \emph{traits} en Scala.

\section{Traits}
\label{sec:traits}

\ejercicio Vamos a modelar un comportamiento común a gatos, leones, tigres, panteras, jaguares, etc. Vamos a definir una interface común (\emph{trait})  \texttt{Felino}. En la figura~\ref{fig:jerarquia-felinos}. Defina el \emph{trait} \texttt{Felino} y las diferentes sub-especies de la super clase ella. 

\begin{figure}[h]
  \centering
  \includegraphics[width=10cm,height=8cm]{imagenes/JerarquiaFelinos}
  \caption{Jerarquía de gatos}
  \label{fig:jerarquia-felinos}
\end{figure}

Tenga en cuenta que:

\begin{itemize}
\item En \texttt{Felino} se encuentra el color.
\item En \texttt{Felino} se encuentra el sonido que hace cada uno de los diferentes tipos.
\item Únicamente \texttt{Gato} tiene una comida favorita.
\item Un \texttt{León} tiene un tamaño de melena.
\item Recuerde que todos los atributos tiene sus correspondientes métodos \emph{getters}.
\item Los constructores se hacen en las clases no en los \emph{traits}.
\end{itemize}

$\square$

\ejercicio Define un \emph{trait} llamado \texttt{Forma} y tiene tres métodos abstractos:

\begin{itemize}
\item \texttt{tamaño} que retorna el número de lados.
\item \texttt{perímetro} que retorna la longitud total de sus lados.
\item \texttt{área} que retorna el área de la forma en cuestión. 
\end{itemize}

Implemente \texttt{Forma} con tres clases: \texttt{Círculo}, \texttt{Rectángulo} y \texttt{Cuadrado}. En cada caso implemente los tres métodos. Asegure un constructor principal con los parámetros para cada forma y que los campos son accesibles (\emph{getters}).

$\square$

\ejercicio La solución del ejercicio anterior produjo tres tipos diferentes de formas. Sin embargo, este no modela la relación entre los elementos correctamente. Un \texttt{Cuadrado} no es solamente una forma también es una forma de tipo \texttt{Rectángulo} donde la longitud y la altura son iguales.

Refactorize la solución del anterior ejercicio así el \texttt{Cuadrado} y \texttt{Rectángulo} son subtipos de un tipo común llamado \texttt{Rectangular}.

\textbf{Note:} Un \emph{trait} puede extender otro \emph{trait}.

$\square$

\section{Sealed Traits}
\label{sec:sealed-traits}

\ejercicio Revisemos la forma implementada en la anterior seccion~\ref{sec:traits}. En primer lugar haga \texttt{Forma} en un \texttt{sealed trait}. Entonces, escriba un objeto único (\emph{singleton object}) llamado \texttt{Draw} con un método \texttt{apply} que tomar una \texttt{Forma} como argumento y returna un descripción de él en la terminal.

Por ejemplo:

\begin{alltt}
scala> Draw(Circulo(10))
res0: String = Un circulo de radio 10.0cm
scala> Draw(Rectangulo(3,4))
res1: String = Un rectangulo de ancho 3.0cm y largo 4.0cm
\end{alltt}

Finalmente, verifique que el compilador se queja cuando se comente una clausula \texttt{case} en el método \texttt{Draw}.

$\square$

\ejercicio Escriba una clase \texttt{Color}

\begin{enumerate}
\item Da al color tres propiedades: RGB (Red,Green,Blue).
\item Cree tres colores predefinidos: \texttt{Rojo}, \texttt{Amarillo} y \texttt{Rosa}.
\item Suministre un medio para que la gente produzca sus propios colores personalizados de \texttt{Color} con sus propios valores de RGB.
\end{enumerate}

Este ejercicio es abierto a interpretación. $\square$


% \section{Bibliografía}
% \label{sec:bibliografia}
% \bibliographystyle{amsalpha}
% \bibliography{S4N_Git_Tutorial}

\end{document}

%%% Local Variables:
%%% mode: latex
%%% TeX-master: t
%%% End:
