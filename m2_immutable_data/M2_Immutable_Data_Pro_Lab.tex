\documentclass[12pt]{article}

\usepackage[utf8]{inputenc}
\usepackage[spanish]{babel}
\usepackage{enumerate}
\usepackage{amsmath}
\usepackage{txfonts}
\usepackage{graphicx}
\usepackage{keystroke}
\usepackage{color}
\usepackage[colorlinks = true,
linkcolor = green,
urlcolor = blue,
citecolor = yellow,
anchorcolor = brown]{hyperref}
\usepackage{listings}
\usepackage{alltt}
\usepackage{multicol}

\definecolor{mygreen}{rgb}{0,0.6,0}
\definecolor{mygray}{rgb}{0.5,0.5,0.5}
\definecolor{mymauve}{rgb}{0.58,0,0.82}

\lstset{ %
  backgroundcolor=\color{white},   % choose the background color; you must add \usepackage{color} or \usepackage{xcolor}
  basicstyle=\footnotesize,        % the size of the fonts that are used for the code
  breakatwhitespace=false,         % sets if automatic breaks should only happen at whitespace
  breaklines=true,                 % sets automatic line breaking
  captionpos=b,                    % sets the caption-position to bottom
  commentstyle=\color{mygreen},    % comment style
  deletekeywords={...},            % if you want to delete keywords from the given language
  escapeinside={\%*}{*)},          % if you want to add LaTeX within your code
  extendedchars=true,              % lets you use non-ASCII characters; for 8-bits encodings only, does not work with UTF-8
  frame=single,                    % adds a frame around the code
  keepspaces=true,                 % keeps spaces in text, useful for keeping indentation of code (possibly needs columns=flexible)
  keywordstyle=\color{blue},       % keyword style
  language=Octave,                 % the language of the code
  morekeywords={*,...},            % if you want to add more keywords to the set
  numbers=left,                    % where to put the line-numbers; possible values are (none, left, right)
  numbersep=5pt,                   % how far the line-numbers are from the code
  numberstyle=\tiny\color{mygray}, % the style that is used for the line-numbers
  rulecolor=\color{black},         % if not set, the frame-color may be changed on line-breaks within not-black text (e.g. comments (green here))
  showspaces=false,                % show spaces everywhere adding particular underscores; it overrides 'showstringspaces'
  showstringspaces=false,          % underline spaces within strings only
  showtabs=false,                  % show tabs within strings adding particular underscores
  stepnumber=2,                    % the step between two line-numbers. If it's 1, each line will be numbered
  stringstyle=\color{mymauve},     % string literal style
  tabsize=2,                       % sets default tabsize to 2 spaces
  title=\lstname                   % show the filename of files included with \lstinputlisting; also try caption instead of title
}

\newcommand{\noterm}[1]{\color{blue} \langle #1 \rangle \color{blue}}
\newcommand{\produce}{\color{red} \coloneqq \color{red}}
\newcommand{\alter}{\color{red} \mid \color{red}}
\newcommand{\catterm}[1]{\color{magenta} \text{#1} \color{magenta}}
\newcommand{\term}[1]{\color{cyan} \text{#1} \color{cyan}}
\newcommand{\openopt}{\color{black} \text{[} \color{black}}
\newcommand{\closeopt}{\color{black} \text{]} \color{black}}
\newcommand{\severals}{\color{black} \ldots \color{black}}

\newcounter{ejercicio}
\newcommand{\ejercicio}{\stepcounter{ejercicio}%
  \paragraph\noindent\textbf{Ejercicio \theejercicio.\hspace{4pt}}}

\newcounter{problema}
\newcommand{\problema}{\stepcounter{problema}%
  \paragraph\noindent\textbf{Problema \theproblema.\hspace{4pt}}}
\newcommand{\WindowsLogo}{\raisebox{-0.1em}{%
    \includegraphics[height=0.8em]{../../imagenes/Windows_3_logo_simplified}}}
\newcommand{\WinKey}{\keystroke{\WindowsLogo}}

\newcommand{\Subversion}{\href{https://subversion.apache.org/}{subversion}}
\newcommand{\Riouxsvn}{\href{https://riouxsvn.com/}{Riouxsvn}}

\title{Taller de Tipos de datos inmutables\\Listas}
\date{25 de febrero de 2021}
\author{S4N Campus}
%\institute{S4N}

\begin{document}
\maketitle

\section{Preliminares}
\label{sec:preliminares}

Este taller tiene como objetivo la creación de funciones recursivas para el manejo de listas.

\section{Coincidencia de patrones}
\label{sec:listas}

La coincidencia de patrones trabaja de forma similar a un
\texttt{switch-case} en Java. La idea básica de la coincidencia de
patrones es que \texttt{match} puede descender en la estructura de la
expresión que examina y extrae sub-expresiones de esa estructura. En primer lugar, si un patrón coincide con un literal, ese valor cumple.

\begin{lstlisting}[language=Scala]
10 match => { case 10 => "Diez" }
\end{lstlisting}

En el anterior caso el valor del literal $10$ en el \texttt{case} coincide con el valor entrado en \texttt{match}. El resultado es \texttt{"Diez"}.

\begin{lstlisting}[language=Scala]
10 match => {}
   case 1 => "No"
   case n => s"Diez = $n"
}
\end{lstlisting}

En el anterior caso, el literal
$1$ no coinicide con el valor entrado en el \texttt{match}, entonces
la variable \texttt{n} se hace coincidir con el valor
$10$ y por lo tanto el resultado final es \texttt{"Diez = 10"}.

Miremos más ejemplos:

\begin{itemize}
\item \texttt{List(4,5,6) match \{ case \_ => 1 \}} resulta en
  \texttt{1}. El \texttt{\_} es un comodín que significa cualquier
  valor.
\item \texttt{List(4,5,5) match \{ case Const(h,\_) => h \}} resulta en \texttt{4}. La coincidencia de patrones hace que el valor del constructor \texttt{Const}
  quede en \texttt{h} y se hace coincidir con dicho valor.
\item \texttt{List(4,5,6) match \{ case Const(\_,t) => t \}} resulta en \texttt{List(5,6)}. La coincindencia de patrones que el valor del constructor \texttt{Const}
  quede en \texttt{t} y se hace coincidir con el resto de la lista.
\item \texttt{List(4,5,6) match \{ case Nil => 1 \}} resulta en \texttt{MatchError}. La lista de \texttt{match} no esta vacía.
\end{itemize}

\ejercicio. ¿Cuál es el resultado de la siguiente expresión \texttt{match}?
\begin{lstlisting}[language=Scala]
val x = List(4,5,6,7,8) match {
   case Const(x, Const(5, Const(7, _)))        => x
   case Nil                                    => 1
   case Const(x,Const(y,Const(6, Const(7,_)))) => x + y
   case Const(h,t)                             => h + sum(t)
   case _                                      => 777
}
\end{lstlisting}$\square$

\section{Operaciones sobre listas}
\label{sec:oper-sobre-list}


\ejercicio Implementa la función \texttt{tail} que remueva el primer elemento de un lista. $\square$
\ejercicio Implementa la función \texttt{head} que devuelva el primer elemento de la lista. $\square$
\ejercicio Implemente la siguiente función.
\begin{lstlisting}[language=Scala]
def and(lst:List[Boolean]):Boolean = ???
\end{lstlisting}
Esta función recibe un arreglo de valores \texttt{Boolean} y devuelve \texttt{true} si todos los valores son verdaderos,
en caso contrario devuelve \texttt{false}. $\square$
\ejercicio Implemente la siguiente función.
\begin{lstlisting}[language=Scala]
def or(lst:List[Boolean]):Boolean = ???
\end{lstlisting}
Esta función recibe un arreglo de valores \texttt{Boolean} y devuelve \texttt{false} si todos los valores son falsos,
en caso contrario devuelve \texttt{true}. $\square$
\ejercicio Implemente la siguiente función.
\begin{lstlisting}[language=Scala]
def max(lst:List[Int]):Int = ???
\end{lstlisting}
Esta función recibe un arreglo de valores \texttt{Int} y devuelve valor máximo de todos los valores en la lista. $\square$
\ejercicio Implemente la siguiente función.
\begin{lstlisting}[language=Scala]
def min(lst:List[Long]):Long = ???
\end{lstlisting}
Esta función recibe un arreglo de valores \texttt{Long} y devuelve valor mínimo de todos los valores en la lista. $\square$
\ejercicio Implemente la siguiente función.
\begin{lstlisting}[language=Scala]
def minMax(lst:List[Double]):(Double,Double) = ???
\end{lstlisting}
Esta función recibe un arreglo de valores \texttt{Double} y devuelve
valor (mínimo,máximo) de todos los valores en la lista. Se deben
utilizar \href{https://docs.scala-lang.org/tour/tuples.html}{tuplas}
para el resultado final.$\square$

\section{Definicio de tipos algebraicos los Naturales}
\label{sec:definicio-de-tipos}

Los \href{https://es.wikipedia.org/wiki/N%C3%BAmero_natural}{números naturales} pueden ser definidos como un tipo de dato algebraico. Donde el caso base se tiene el \texttt{Cero} y el caso recursivo es el sucesor de un natural. En Haskell se definen de la siguiente manera.

\begin{lstlisting}[language=Scala]
data Nat = Cero | Suc Nat
\end{lstlisting}

  Donde \texttt{Nat} es el tipo. \texttt{Cero} y \texttt{Suc} son los constructores de valores.

  \ejercicio Implemente los \texttt{Nat} en Scala. Tenga en cuenta la siguiente iteración.

\begin{alltt}
scala> val cero = Cero
val cero: co.s4n.inmutable.nat.Nat.type = Cero
scala> val uno = Suc(Cero)
val cero: co.s4n.inmutable.nat.Suc = Suc(Cero)
scala> val dos = Suc(Suc(Cero))
val cero: co.s4n.inmutable.nat.Suc = Suc(Suc(Cero))
\end{alltt}

  $\square$

  \ejercicio Implemente la función \texttt{fromNatToInt} que toma un número natural \texttt{Nat} y lo transforma a su valor \texttt{Int}. $\square$.
  \ejercicio Implemente la función \texttt{fromIntToNat} que tomar valores enteros
  positivo (inclusive le cero) y produce el correspondiente número natural. $\square$


% \section{Bibliografía}
% \label{sec:bibliografia}
% \bibliographystyle{amsalpha}
% \bibliography{S4N_Git_Tutorial}

\end{document}

%%% Local Variables:
%%% mode: latex
%%% TeX-master: t
%%% End:
